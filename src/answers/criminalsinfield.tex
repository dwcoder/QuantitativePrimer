\begin{answer}{criminalsinfield}
  The solution to this question requires many assumptions on the rationality of the criminals which probably don't hold in practice.
  You should also ask some further questions to establish that you are allowed to address the criminals before they make a run for it, which probably means you have a very loud voice or a megaphone.
  Good interviewers will tell you this as part of the question, while others might want you to prod.

  Start with $n=1$.
  \index{tricks!start with simplest case}
  Were there a single murderer, you would shoot him if he tried to escape.
  Because he would know the probability of survival is zero, he wouldn't try.
  Were there only two murderers in the field, each would have a probability of surviving of 0.5, and thus would attempt to escape.
  But, if you told one of the two that you would definitely shoot \emph{him} if the two of them attempted to escape, he wouldn't make an attempt, bringing you back to the situation with a single murderer.
  You can generalise this outcome to three or more murderers, but for any given group you need to identify one murderer as the so-called ``sacrificial escapee'', and make him aware of it.
  The easiest way is to number them from $1$ to $n$, and tell them that, for any group that attempts an escape, the member with the largest number will be shot.
  By induction, no group will attempt an escape.
  If you don't have time to number them, the go-to solution is to tell the murderers that, in the event of an attempted escape, the tallest member of the group will be shot.

  Though this is an illogical question, try not to get frustrated and point this out to the interviewer.
  It is asked in the same spirit as questions such as ``how many ping pong balls can fit into an Airbus A380,'' or ``how many ties are sold in London in a year.''
  The interviewer wants to see your thought process.
  Through all of my interviews I only got a question like this once and while I think they are rare in quantitative interviews, I believe they are more prevalent in investment banking interviews in general.
  I would suggest you look up two or three of these, and study the format of the solutions.
\end{answer}
