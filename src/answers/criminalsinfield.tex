\begin{answer}{criminalsinfield}
  The solution to this question requires many assumptions on the rationality of the criminals which probably don't hold in practice.
  You should also ask some further questions to establish that you are allowed to address the criminals before they make a run for it, which probably means you have a very loud voice or a megaphone.
  Good interviewers will tell you this as part of the question, others might want you to prod.

  Use the trick where we start with $n=1$.
  \index{Tricks!Start with simplest case}
  If there is just one criminal, we would shoot him if he tries to run.
  He knows his probability of surviving is 0, so he will not run.
  If there are two criminals each will have a probability of 0.5 to survive, and thus they will attempt an escape.
  But if you tell one of them you will definitely shoot \emph{him} if they attempt to run, he won't run.
  Then we are back to the situation with one criminal.
  We can generalise this to three or more criminals, but we will soon realise that for any group, we need to identify one specific prisoner as the sacrificial runner and make him aware of it.
  The easiest way is to number them from $1$ to $n$ and tell them that for any group who runs, the one with the highest number will get shot.
  By induction, no group will start to run.
  If you don't have time to number them, you can just tell them you will always shoot the tallest one in any group that runs.

  Since this is such a frustratingly illogical question, the last thing you should do is get frustrated and point out how illogical the question is.
  It is asked in the same spirit as questions such as ``how many ping pong balls can fit into an Airbus A380,'' or ``how many ties are sold in London in a year.''
  The interviewer wants to see your thought process.
  Through all of my interviews I only got a question of this nature once and while I think they are rare in quantitative interviews, I've believe they are more prevalent in investment banking interviews in general.
  I would suggest you look up two or three of these, and study the format of the solutions.
\end{answer}
