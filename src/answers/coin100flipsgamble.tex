\begin{answer}{coin100flipsgamble}
This can be solved by using the Central Limit Theorem, or the normal approximation to the binomial distribution.
\index{tricks!Central Limit Theorem}
You have
\begin{align*}
  Y \sim \text{Binomial}(p=0.5, n=100)
\end{align*}
and for large $n$ you can approximate the binomial distribution through the normal distribution,
\begin{align*}
  Y &\stackrel{\sim}{\text{\tiny approx}}  \text{Normal}(np, np(1-p))
  \text{.}
\end{align*}
The values of $n$ and $p$ in the problem give
\begin{align*}
  np      &= 100\left(\nicefrac{1}{2}\right)    = 50\\
  np(1-p) &= 100\left(\nicefrac{1}{2}\right)^2  = 25
  \text{,}
\end{align*}
resulting in
\begin{align*}
  Y & \sim
  \text{Normal}\left( 50, (5)^2 \right)
  \text{.}
\end{align*}
In order to evaluate whether the gamble is worthwhile you have to calculate its expected return:
\begin{align*}
  \E(\text{Gamble}) &= -1 + (10)P(\text{Win})  + (0)P(\text{Lose}) \\
                    &= -1 + (10)P(Y>60)
                    \text{.}
\end{align*}
Given that you are in an interview, you need to evaluate $P(Y>60)$ without a calculator.
\index{tricks!normal probabilities in your head}
With the normal distribution, approximately 95\% of the data lies within two standard deviations of the mean.
For the current problem, two standard deviations from the mean is conveniently located at 60, as the following graph indicates:
\begin{center}
  \includegraphics[width=0.8\textwidth]{./plots/prettynorm/prettynorm.pdf}
\end{center}
The probability of interest is $P(Y>60 = 0.025)$ and thus the expected value of the gamble is
\[
  \E(\text{Gamble}) = -1 + (10)(0.025) = 1 - 0.25 = -0.75
  \text{.}
\]
On average you lose 75 pence for playing the game so it is not worthwhile.
The fair price, denoted $\pounds_{\text{fair}}$, occurs where the gamble breaks even:
\begin{align*}
  \E(\text{Gamble})     &=  0    \\
  \pounds_{\text{fair}} - 0.25 &=  0    \\
  \pounds_{\text{fair}}        &=  0.25
\end{align*}
and this is the highest price at which a
rational and risk-neutral person should start agreeing to play.
The assumptions of rationality and risk neutrality are very important.
If a billionaire gets immeasurable physical pleasure from engaging in risky gambles---and therefore is not risk neutral---the rational course of action is probably to spend all of their excess money on unfair gambles.
Conversely, risk-neutral people might not always act rationally.

\end{answer}
