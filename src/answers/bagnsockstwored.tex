\begin{answer}{bagnsockstwored}
Let $R_i$ indicate the event where the $i$th sock is red.
Let $r$ be the number of red socks in a bag of $N$ socks.
We need
\begin{align*}
  P(R_1,R_2)     &= \frac{1}{2} \\
  P(R_1)(R_2|R_1) &= \frac{1}{2} \\
  \frac{r}{N} \times
  \frac{r-1}{N-1}
   &=
  \frac{1}{2}
  \text{.}
\end{align*}
Use trial and error.
\index{Tricks!Start with simplest case}
If $N=2$ both socks will be red with certainty, so this cannot work.
When $N=3$ we have
\[
  \frac{r}{3} \times
  \frac{r-1}{2}
   =
  \frac{r(r-1)}{6}
   =
  \frac{1}{2}
\]
and there aren't any integer values of $r$ that would make this hold.
For $N=4$ we have
\[
  \frac{r}{4} \times
  \frac{r-1}{3}
   =
  \frac{r(r-1)}{12}
   =
  \frac{1}{2}
\]
and this works for $r=3$.
Therefore the smallest possible value of $N$ for the stated scenario is four.

\end{answer}
