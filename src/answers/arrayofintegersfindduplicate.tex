\begin{answer}{arrayofintegersfindduplicate}
 Without sorting, we have to keep track of the numbers we have seen.
 You could mention using a hashtable, or some other container with key-value pairs.
 The Python \verb+dict+ is an example.
 That way we can quickly check whether each number is already in the dictionary, and stop looking if we find one that is.
 If the number isn't in the dictionary, add it.
 The Python \verb+defaultdict+ will be even better, as it handles the key error when the key is not yet in the dictionary.
 Another way to do it is to add the values into a list in such a way that the list remains sorted.
 For each integer, you then do a binary search on the list to see whether the integer is already there.
 This is the basic idea of a binary tree.
 Think of a few ways to do this using various containers in your favourite language, and think of the benefits of each of them.
 This question has many solutions, and there are no wrong answers (unless the interviewer is very opinionated and prejudiced against your container of choice).


 With sorting it is trivial.
 Sort the numbers, go through them one by one, and ask ``Is the next value equal to the current value?''
 As soon as the answer is ``yes'' we can stop the search.
 The question about your favourite sorting algorithm is a loaded one.
 The interviewer is likely to ask you to explain how it works, about its complexity, and to write it in actual or pseudocode.
 Therefore, ensure you do have a favourite sorting algorithm and make sure you know its details.
 Also ensure your favourite isn't Bubblesort.
 It is inefficient and acts only as an example to get new programmers thinking about algorithms.
 It is also overused by people doing interviews.
Your favourite sorting algorithm should be Mergesort, for the following reasons.
\index{Mergesort}
 \begin{enumerate}
   \item Mergesort is efficient. It has $O(n \log n)$ complexity, which is on par with the best sorting algorithms used in production.
   \item Mergesort is actually used in production, albeit usually as part of more efficient algorithms such as Timsort. The toy-example algorithms like Bubblesort are only used in classroom examples (and by interviewees).
   \item Due to its divide-and-conquer strategy it can easily be parallelised, which has obvious benefits when used with computing clusters and GPU programming.
   \item It allows you to show that you understand recursion. Bonus points.
   \item For all its strengths, it is surprisingly simple to code. The main workings require only six lines of code---as shown in the example Python code below.
 \end{enumerate}
The basic idea of Mergesort is to split the list into two parts, sort each part, and combine them back into single, sorted list.
Since the second step requires sorting, we can do it recursively.
The workings are given in the below pseudocode, were we assume that we have a function called
\verb+merge_sorted_vectors+,
which receives two sorted vectors as arguments and combines them into one sorted vector.

\begin{algorithmic}[1]
  \Statex
  \Function{MergeSort}{$\vec{x}$}
  \State
        $n$ = length($\vec{x}$)
    \If{n is 1}
      \State
      \Return $\vec{x}$
      \Comment{ Recursion base case}
     \EndIf
  \State
  $m$ = round($\nicefrac{n}{2}$) \Comment Vector middle index
  \State
  firsthalf = $\{x_1, x_2, \ldots ,x_m\}$
  \State
  secondhalf = $\{x_{m+1}, x_{m+2}, \ldots, x_n \}$
  \State
  firsthalf\_sorted = MergeSort(firsthalf)
  \Comment Recursive call
  \State
  secondhalf\_sorted = MergeSort(secondhalf)
      \State \Return{ MergeTwoSortedVectors(firsthalf\_sorted, secondhalf\_sorted) }
  \EndFunction
\end{algorithmic}
The code is simple since most of the tricky workings is hidden in the MergeTwoSortedVectors() function.
It is still, however, easier to write than Bubblesort.

The Python implementation of Mergesort below is even easier to read than the pseudocode.
The function has only six lines of code to remember, which you can reduce further at the cost of readability.
\begin{minted}[samepage=true]{python}
def mergesort(unsortedlist):
    if (len(unsortedlist)==1):
        return unsortedlist

    mid = len(unsortedlist)/2
    firsthalf = unsortedlist[0:mid]
    secondhalf = unsortedlist[mid:]
    return merge(mergesort(firsthalf), mergesort(secondhalf))
\end{minted}
Here is one way to do the merge function.
\begin{minted}[samepage=true]{python}
# These two lists will already be sorted from small to large
def merge(list1, list2):
    sortedlist = []
    while ((len(list1)>0) or (len(list2) > 0)):
        if (len(list1)==0):
            return list2 + list(reversed(sortedlist))
        elif (len(list2)==0):
            return list1 + list(reversed(sortedlist))
        else:
            if (list1[-1] >= list2[-1]):
                sortedlist.append(list1.pop())
            else:
                sortedlist.append(list2.pop())

\end{minted}
\end{answer}
