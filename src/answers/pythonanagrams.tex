\begin{answer}{pythonanagrams}
The interviewer will likely explain what an anagram is before asking you this.
Two words are anagrams if they have the same set of letters, like \emph{toned} and \emph{noted}.
You can re-arrange the letters of the one to spell the other.
The easiest way to check whether two strings are anagrams is to sort them and check whether they are the same.
We can catch a trivial case at the beginning of the function: if the strings don't have the same length, they are definitely not anagrams.
\begin{minted}{python}
def IsAnagram(string1, string2):
    if len(string1) is not len(string2):
      return False

    return sorted(string1) == sorted(string2)
\end{minted}
%
To do this without sorting requires a trick.
We assign
the integers 1--26
to
the letters A--Z.
Then we loop over the two strings, and we add up all the letters in the first string, and subtract all the letters of the second string.
If the two strings are anagrams the final answer will be zero, since every number will be added and subtracted from the total exactly once.
Python 2.7 actually has the function \verb+ord+ that converts characters into their ASCII equivalent numbers, so we can use that.
\begin{minted}{python}
def IsAnagramNoSort(string1, string2):
    if len(string1) is not len(string2):
      return False

    sum = 0

    for letter in string1:
        sum += ord(letter)
    for letter in string2:
        sum -= ord(letter)

    return (sum == 0)
\end{minted}

If the strings are long, or if this is an operation you have to do often, the bester version is one without sorting.
Sorting is expensive: the best sorting algorithms have $O(n\log{n})$ complexity.
For the algorithm that doesn't sort, we only need to access each of the letters once so the complexity is $O(n)$.
\end{answer}
